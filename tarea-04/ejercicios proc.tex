\documentclass{article}
\usepackage[utf8]{inputenc}
\usepackage{subcaption}




\title{Ejercicios PROC}
\author{Daniel Antonio Quihuis Herandez }
\date{12 de octubre del 2022}

\begin{document}

\maketitle

\section{Ejercicio 3.20[*]}
\textbf{In PROC, procedures have only one argument, but one can get the
effect of multiple argument procedures by using procedures that return other procedures. For example, one might write code like}


\begin{lstlisting}
let f = proc (x)

\hspace{1.5cm}  proc (y)...

   \hspace{2cm} in ((f 3) 4)
\end{lstlisting}

\textbf{This trick is called Currying, and the procedure is said to be Curried. Write a Curried
procedure that takes two arguments and returns their sum. You can write x + y in our language by writing −(x, −(0, y)).}

\begin{lstlisting}
    let function-name = 
    
     \hspace{0.5cm} proc (x) 
     
     \hspace{1cm} proc(y) 
     
     \hspace{1.5cm} -(x, -(0, y))
    
\end{lstlisting}
\section{Ejercicio 3.27[*]}
\textbf{Add a new kind of procedure called a traceproc to the language.
A traceproc works exactly like a proc, except that it prints a trace message on
entry and on exit.}

\textit{;; Sintaxis concreta } 

\hspace{1cm} Expression ::= trace-procedure(ID) Expression 

\textit{;; Sintaxis abstracta } 

\hspace{1cm} (trace-proc-exp var body) 

\textit{;; Semantica } 

\hspace{1cm} (value-of (trace-proc-exp var body) $\rho$)) = (trace-proc-val (procedure var body $\rho$))


\end{document}
